\chapter{Podsumowanie i wnioski końcowe}

Zrealizowany projekt pozwala wysnuć wnioski, iż oprogramowanie ROS wraz z dodatkiem MoveIt, mimo pewnych ograniczeń w dobrym stopniu nadaje się do sterowania robotem. W czasie przeprowadzanych eksperymentów możliwe było zaobserwowanie, iż w pewnym momencie dochodziło się do swoistych fizycznych ograniczeń związanych z konstrukcją samego robota (powstały jako wydruk 3D), przez co niemożliwe było uzyskiwanie chociażby dokładniejszych sposobów pozycjonowania. Niemniej projekt miał na celu bardziej przetestować funkcjonalność, możliwości kontroli urządzenia, co też udało się w pełni. 

Jeżeli chodzi o samo oprogramowanie ROS to istotnie znacząco ułatwia ono i przede wszystkim standaryzuje kwestie związane ze sterowaniem budowanymi robotami. Przy czym niezbędne jest poświęcenie sporej ilości czasu na jego zainstalowanie, uruchomienie i skonfigurowanie. Dla wielu potencjalnych użytkowników może to stanowić barierę zniechęcającą do zapoznawania się z tym środowiskiem, ze względu na ograniczoną społeczność innych programistów oraz nierzadko niepełną dokumentację, przez co wielokrotnie rozwiązania wielu napotkanych problemów należy poszukiwać samodzielnie, co nierzadko zajmuje całe tygodnie. Podobnie było przy realizacji tego projektu, gdzie trudności stanowiło już zainstalowanie samego ROSa, ze względu na potrzebę rozwiązania konfliktów przy buildowaniu plików. Tak zatem w początkowej fazie użytkowania oprogrmowania ROS większość czasu poświęca się na rozwiązywaniu występujących błędów, aniżeli samym pisaniu kodu.

W czasie realizacji projektu wyszły również na jaw kwestie związane z fizycznymi ograniczeniami posiadanego robota. Jako, iż został on w całości wydrukowany a części jakie użyto do jego budowy należą raczej do rozwiązań nieprzemysłowych, toteż cała konstrukcja wyróżniała się dosyć sporą zawodnością. Przede wszystkim przewody do prototypowania, którymi łączono sporo elementów elektroniczynych, pinów mikrokontrolera z czujnikami, sterownikami silników krokowych nierzadko nie spełniały swoich funkcji. Przez przypadek wypadały, nie łączyły jak powinny. W pewnym momencie prototyp robota był na tyle złożony, iż chcą ewentualnie w przyszłości dalej rozwijać konstrukcję, konieczne by było wymienić zawodne elementy - takie jak chociażby wspomniane przewody do prototypowania na solidniejsze połączenia.

Podobnie kwestia miała się z faktem luzów robota obecnych na poszczególnych złączach. Fakt, iż robot został wydrukowany, również niektóre jego przekładnie napędowe powstały w ten sposób, powoduje iż poszczególne osie charakteryzują się sporymi luzami na poziomie jednego bądz nawet dwóch stopni. Przez to trudno mówić o wyjątkowej dokładności urządzenia. O ile w przypadku rozpatrywanego robota nie stanowiło to większe problemu, gdyż powstała platforma miała formę eksperymentalną, edukacyjną, o tyle dla bardziej profesjonalnych zastosowań niezbędne byłyby znacznie precyzyjniejsze rozwiązania. Niestety też znacznie droższe. 

Należy również wspomnieć o wpływie grawitacji na ruch złącz na które siła ta działa. Dało się zauważyć, iż ramię drugie robota, w przypadku którego wpływ grawitacji był największy poruszało się z różną prędkością w zależności od kierunku ruchu. Tym samym, przy opuszczaniu ramię przekraczało zadaną pozycję. W przeciwnym kierunku, przy podnoszeniu ten problem nie występował.

Nie da się także nie zauważyć ograniczeń konstrukcyjnych jakie posiadała konstrukcja robota wykorzystanego w projekcie. Warto tutaj wymienić takie szczegóły jak niewielki zakres ruchu poszczególnych ramion, a także niewielką liczbę stopni swobody. Wydaje się iż szósta bądź nawet siódma oś pozwalałaby na zwielokrotnienie możliwości konstrukcji.

Na temat samego chwytaka również nasuwają się pewne wnioski. Z prowadzonych eksperymentów wynikało, iż człon chwytający powinien posiadać pewną formę sprzężenia zwrotnego związanego z siłą chwytu. Problematyczne też okazało się sterowanie chwytakiem poprzez współczynnik wypełnienia. Niestety ale przewody zasilające, biegnące bez ekranowania w pobliżu silników krokowych były narażone na zakłócenia. 

Dodatkowo posiadany robot posiadał zbyt mało stopni swobody by móc realizować zadanie nullspace exploration - czyli sytuacji kiedy to końcówka, kiść robota nie pozostaje w tym samej pozycji natomiast całe ramię wykonuje swoisty ruch okrężny. \cite{latex_code}